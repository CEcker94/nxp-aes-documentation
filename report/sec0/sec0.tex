%*****************************************************************
%*************************** Section 0 ***************************
%**************************** Vorwort ****************************
%*****************************************************************


\pagestyle{fancy}
\rhead{\thepage} \chead{} \lhead{\nameref{Sec0}}
\cfoot{}


\addcontentsline{toc}{section}{Vorwort und Projekt-Links}
\section*{Vorwort und Projekt-Links}\label{Sec0}

Dieser Bericht umfasst die bei der Durchführung des Projekts \glqq{}Aufbau und Programmierung eines Modellfahrzeugs für den  \ac{NXP}-Cup\grqq{} über den Zeitraum von zwei Semestern gesammelten Erfahrungen und Ergebnisse. Wir als Projektteam bedanken uns herzlich bei Herrn Prof. Dr. Rausch und Herrn Arne Kullina für die Unterstützung bei diesem Projekt.\vspace{11pt} 

Wir freuen uns außerdem darüber, dass wir bei der Innovation Challenge von electromaker.io den dritten Platz mit unserem Projekt belegen konnten. Da Electromaker ein Partner von NXP ist, muss jedes Team ihr Projekt auf der Plattform electromaker.io veröffentlichen, um am \ac{NXP}-Cup teilnehmen zu können. Besonders gut dokumentierte oder interessante Fahrzeug-Projekte können bei der Innovation Challenge im Vorfeld des \ac{NXP}-Cups ein Preisgeld von bis zu 1000€ gewinnen.\vspace{11pt}

\begin{center}
\textbf{Unsere Projektdokumentation auf electromaker.io}\\
\textbf{findet sich unter diesem Link:}\vspace{9pt} 

\url{https://www.electromaker.io/project/view/fast-and-furious}

\end{center}

Sowohl die gesamte Dokumentation (Stromlaufpläne, Bilder, Pinbelegungen, 3D-Druck-Dateien, Datenblätter, ...), als auch den Programmcode, haben wir auf GitHub veröffentlicht. Wir hoffen, dass das nächste Team der HAW Landshut (Hochschule für angewandte Wissenschaften Landshut) auf unsere ausführlich dokumentierte Grundlage aufbauen und den NXP-Cup gewinnen wird.\vspace{11pt} 

\begin{center}
\textbf{Der Programmcode und die Projektdokumentation}\\
\textbf{finden sich unter diesen Links:}\vspace{9pt} 

\url{https://github.com/CEcker94/nxp-aes-proj}\vspace{5pt} 

\url{https://github.com/CEcker94/nxp-aes-documentation}

\end{center}


\newpage